\subsection{Navigation in Location-based Games}
As seen in the examples mentioned earlier, location-based games (LBGs) utilize points of interest (POIs) in their gameplay, which brings up the requirement of navigating between POIs, when the games take place in cities. This brings up opportunities to gain additional knowledge of the city, and not solely at the POIs. The potential of getting familiar with the city while walking may not be fully utilized, since LBGs often revolve around POIs rather than what is between. Previous studies revolving around the navigational aspect within LBGs is limited. 
Gordillo et al. made a hybrid LBG in the city for tourists\cite{Learninggamified}. The game offered three POIs which were marked on a 2D map, requiring the participant to go there in order to trigger activities provided at the location.  One distance required travelling 3 km (from G\"uell Park to Casa Batll\`{o}), bringing the game to a pause until arrival at the point of interest.  The outcome of the study is unknown, as no test was carried out. From this, we assume that the navigation mainly served as a requirement for leading the player from one POI to another and not as a part of the game activities.
%From this, we observe that the navigation served minimal priority, but merely serves as a requirement for leading the player from one point of interest (POI) to another. 

Several LBGs have used 2D maps with Global Positioning System (GPS) technology (e.g. google maps) in a city related context, in order to guide their participants to POIs \cite{TheoreticalAndMethod, Learninggamified, knowcity, Carrigy:2010:DEP:1868914.1868929, GamingTourism, Procyk:2013:GLG:2468356.2468550, Bell:2009:ESN:1518701.1518723}. To the best of our knowledge, no 2D maps have integrated game activities such as those that are found at the POIs. Therefore, we assume that game activities such as answering questions about the physical space and gaining points either disappear or serve no purpose until the arrival to the next location. Furthermore, we have not been able to find any studies that investigate or evaluate whether navigating with a 2D map is preferable in the context of LBGs.
%2D maps are non-coherent to the rest of the gameplay, and continuously disrupts the game experience. The 2D map does not promote any interplay between the physical and virtual domain, and game mechanics (e.g. getting points, puzzles) either disappears or serves no purpose until arrival to next location.  To the best of our knowledge, no studies have investigated or evaluated whether navigating with a 2D map is preferable in the context of location based games. 

We have investigated the use of navigation in several LBGs, in terms of the interplay between the physical and virtual domain, use of ludic and pedagogic elements, and whether it is supported by a narrative. Some LBGs revolve around progressing a story. These types of games depend on sound, and do not depend on visuals for navigating, such as in Blyte et al. Events offered in these games are triggered based on how the player chooses to navigate, giving navigation a crucial role in the overall experience\cite{InterdisciplinaryCriticism}.

In Riot!\cite{InterdisciplinaryCriticism}, players navigated freely in a restricted area. However, its design may only be appropriate in a small bounded area due to the extended freedom of exploration, and could be problematic if transferred to a wider context (e.g. an entire city) due to longer distances between POIs. Epstein and Vergani made a similar study on a walking tour in the city Venice, which likewise incorporated the narrative space into the navigation, but instead kept a more linear narrative structure \cite{MobileTechnologies}. A narrator in the application verbally explained where to make turns, and at the same time made comments on the physical environment. The outcome of the study did not reveal the users' experiences concerning the navigation.
%A qualitative study made by Blythe et al. investigated the enjoyability of a location based game revolving around progressing a story\cite{InterdisciplinaryCriticism}. The participants navigated freely in a restricted area, and the story changed dynamically in relation to their location. In interviews, the participants stated that they found the experience enjoyable in relation to them having control of the story. The game highly promoted interplay between the physical and virtual domain, but its design may only be appropriate in a small bounded area due to the extended freedom of exploration, and could be problematic if transferred to a wider context (e.g. a city) due to longer distances between POIs. Epstein and Vergani made a similar study on a walking tour in the city Venice, which likewise incorporated the narrative space into the navigation, but instead kept a more linear narrative structure \cite{MobileTechnologies}. A narrator in the application verbally explained where to make turns, and at the same time made comments on the physical environment. The outcome of the study did not reveal the users' experiences concerning the navigation.

Both Blythe et al. and Epstein and Vergani encourage the user to explore, but only in relation to the person handling the application due to the use of headphones. Our context deals with families, which would require sharing information. Utilizing audio without it being communicated through headphones would be problematic in terms of navigating in areas with many sounds. 

Eguma et al. devised a LBG for tourists utilizing a sightseeing navigation system to promote awareness of surroundings and enjoyability\cite{HideAndSeek}. The authors proposed creating a navigational system using augmented reality (AR) to display descriptive information from air tags and upon arrival, the participants would have to seek out a character in the surroundings. The concept does however make use of a map, in terms of leading the participants to the area requiring AR for navigating. The aim of the system was letting the user become aware of the surroundings, using 'benefit of inconvenience', which is the idea of something being inconvenient to find, increasing the desire of finding it. The authors did not conduct a study, and therefore the outcome is unknown.

Utilizing AR combined with physical props has served as the navigational method in some LBGs. Morrison et al. conducted a comparative study on a technique called Maplens involving displaying location information on a physical map using augmented reality, comparing it to a 2D map with incorporated accessibility to read about locations, known as DigiMap \cite{Morrison}. This technique was investigated in relation to Flow, Presence and Intrinsic Motivation (IMI). The MapLens had significantly lower scores than DigiMap in most of the questions concerning Flow, Presence and IMI, but its potential was revealed in terms of social interaction since the MapLens encouraged collaborative behaviour.  Morrison et al. found that MapLens did not support ‘playing by moving’, due to its demands of effort, forethought and planning. This behaviour is supported by the study made by Kuikkaniemi et al., which compared MapLens and navigating by following QR codes \cite{LostLab}. The authors did not find MapLens particularly useful based on observations on the participants. The authors observed that the participants rarely used MapLens, and had technical difficulties in terms of the GPS displaying their correct position. The QR codes were a fun way of navigating both indoors and outdoors, based on non-significant observations, but with no concrete examples on why. The QR codes did not promote any environmental awareness, making the interplay between the physical and virtual domain weak. 

As mentioned earlier, hybrid LBGs require a strong interplay between the physical and virtual spaces, supported by game activities and a narrative with the goal of creating an enjoyable learning experience. Based on the above findings in our research, no LBGs have integrated the requirements for a hybrid LBG into the navigation between POIs without relying on sound through headphones, thereby not being suitable for groups of players. For this reason, a new navigational method that is suitable for groups of people, which in our case is families, and that has the potential of integrating both the physical and virtual spaces, is needed. In the following, we describe a navigational method, which we assess to follow the requirements just described. Note that the following is not described in the context of LBGs as it has not been used in other LBGs to the best of our knowledge, in opposition to the other navigational methods described in this section.

Wayfinding using \textit{landmarks} is a navigational method in which objects or structures that mark a locality are used as points of reference, and it is typically used in the communication of route directions\cite{landmarks}. Route directions provide procedures and descriptions that help people build mental representations of the environment they are about to traverse. When following a route, landmarks can be used for re-orientation at decision points such as road intersections and are known as \textit{local} landmarks. Landmarks can also be used for confirming if people are on the right path, known as \textit{route marks}. Finally, landmarks can be used for overall navigation, known as \textit{distant} landmarks. Landmarks can be described by their \textit{saliency}, which defines how much a landmark stands out from the surrounding objects in its environment. Different types of landmarks have different types of saliency. Sorrows and Hirtle categorize landmarks as either \textit{visual}, \textit{cognitive}, or \textit{structural}\cite{landmarksChar}. The saliency of visual landmarks can be characterized by their visual contrast to surrounding objects, e.g. based on the size, shape, position or age of a landmark. For \textit{Cognitive} landmarks, the saliency depends on the meaning of the landmark, e.g. due to the landmark being culturally or historically important. The saliency for structural landmarks depends on the accessibility of the landmark, e.g. the amount of locations a landmark is visible from.  

As wayfinding using landmarks is a navigational method that uses objects in the environment, we see potential in using it in combination with game activities between POIs for LBGs due to its inclusion of objects in the physical space. This could result in a stronger interplay between the physical and virtual spaces during navigation between POIs in LBGs. Furthermore, using landmarks is based on vision instead of sound, indicating that it might be suitable for a group experience. Therefore, we see potential in using landmarks in combination with game activities as the navigational method between POIs for a LBG targeted families. As mentioned earlier, LBGs have a tendency of using 2D maps and GPS for navigation between POIs, however to the best of our knowledge, no LBGs have used landmarks for navigation between POIs. We set out to investigate the enjoyability of using landmarks in combination with game activities for navigation between POIs with the following research question:

\emph{How does landmark navigation in combination with game activities between POIs affect the enjoyability of a location-based game experience for families?}

%Previous research shows a tendency of integrating 2D maps with GPS into location based games, but whether this method is preferable in a game context, is to our best knowledge unknown.  The demands for a hybrid LBG lies in the necessity of a strong interplay between the physical and virtual domain, containing ludic and pedagogic elements and is supported by a narrative.  To our best knowledge, no location based game makes use of all these factors into their navigational method, making the navigation seem less prioritized. 

