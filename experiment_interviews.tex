\subsection{Interviews}
From the interview data, we found that 5 out of 22 children expressed preference towards using maps. One parent mentioned she also preferred map, saying \textit{"It is just always fun to follow a map"}. This parent explained that they were unsure about what to do during the riddle-based navigation and could not remember what they had been told during the instructions. 

Another parent mentioned that map was easy and did not make one aware of the surroundings, because the focus was on walking. In general, we found different opinions on whether the different navigational methods made people aware of the surroundings. One parent clearly stated the the children were not interested in the map at all. In line with several other test participants this parent expressed that it was fun to notice things in the environment they usually do not notice when walking by, making it an optimal method for tourists. Opposed to that opinion, another parent expressed that she focused more on navigating than noticing things in the environment. With riddles, one parent felt that the attention was on the next location to go to, while the map made the participant more aware of the city, because there was more time to look around in the surroundings. 

When asked if they would use riddles as navigational method, if they were tourists in another city, all test participants agreed and answered yes. Some thought it would be a more fun way of learning the city, finding the way and that it would make it possible to see the city in a different way. However, interview data also clearly revealed that several participants would have enjoyed it more, if the riddles were about more interesting landmarks that gave the possibility to learn more e.g. about the city. Preferably this should be done with the children in mind and a few parents proposed a system that can be adjusted depending on, whether they had any children and adjusted content according to the children’s age. Riddles as navigational method was described as \textit{"fun if you have time for it"} by one parent. This reflected the results from the questionnaires. 

The most used word to describe riddle-based navigation was "fun" (11 of 29 words). Other words included exciting, challenging, different, educational and inspiring. A number of participants thought it was fun to answer the questions after the riddles, particularly one child mentioned that it was fun to be able to answer correctly to questions. Another also expressed that it was fun to find the matching pictures in the environment. 

One parent mentioned that the fun part in the riddle-based navigation was to help each other and agree on what they have seen in the environment. Several parents had a similar opinion and stated that they enjoyed collaborating and discussing the answers with the other family members. One parent said it was fun with riddles, 

\begin{quote}
    "(...) because there was something to discuss. Of course you can also discuss what way to go with the map, but that just gave a different experience."
\end{quote}

In terms of group dynamics, it was mentioned that primarily the one with the device was in control, making it a less collaborative experience. One parent mentioned that they collaborated more, when navigating using riddles and not as much with the map. In order to make it more collaborative, one of the participants suggested making the riddles more difficult, encouraging the participants to help each other. This statement supports the experience of another parent, who mentioned that they only collaborated when there was any doubt, otherwise they just followed the child, who was mostly in charge of the device. The interview data revealed a tendency to let the child control the device, which was described as following by one parent,  

\begin{quote}
    "Then one find out about something and the other find out about something else. I 
felt I gave much of the control to Mikki (the child), because I wanted him to think it 
was fun."
\end{quote}
 
In one family, the parent stated that it was much more fun for them both, when the child had the device, because the child was better at using the map and tablets in general.