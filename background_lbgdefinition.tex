%A body of research has focused on using mobile technologies to create game experiences in the context of museums and cities. Previous studies concerning engaging children and families have looked into experiences inspired by treasure hunts, where the players search for written or visual clues in order to find specific items in a museum exhibit \cite{Lynge} \cite{larsen2014tourist}. Jensen investigated, how children can be motivated to engage in a joyful museum experience, by interacting with an agent and taking pictures of art works on a tablet device \cite{Lynge}. Similarly, Larsen \& Svabo investigated treasure trails in pamphlets, where children were dependent on their parents reading out the questions, interpreting the answers and writing them down, making it a family-activity rather than a child-activity \cite{larsen2014tourist}. Since much tourism is about being together and having time with one’s family \cite{larsen2014tourist}, these type of activities are often compelling for tourist families. 

%Mobile devices are an ideal platform to use in this context, because they are increasingly becoming popular among families, as mentioned by Jensen \cite{Lynge}. In this study, we address these experiences and refer to them as \textit{mobile location-based games} (LBGs), as they make use of the physical and virtual spaces to create enjoyable game experiences. Upscaling such experiences at museums to the city context, we did not find any studies on LBGs targeted tourist families. In this context, we found several LBGs for other target groups, where the mobile device is used for interaction at points of interest (POIs), similar to those in museums, e.g. getting information about artefacts, interacting with them or taking pictures as typical behaviours of tourists. 

Avouris \& Yiannoutsou reviewed fifteen LBGs and categorized them as either games designed for player enjoyment (ludic), education (pedagogic) or a combination of both (hybrid) \cite{LBG_Review}. Most of the LBGs at museums, where e.g. children interact with museums exhibits fell under the hybrid category. The authors found that LBGs take place in a \textit{physical space} (e.g. going to a specific physical location) and require some interaction by the player in the \textit{virtual space} (e.g. doing riddles/puzzles, interacting with an avatar or following a map). This results in an interplay between the physical and virtual space, creating what is known as the game space/narrative space \cite{LBG_Review}. They also found that narrative was an underlying element in all LBGs \cite{LBG_Review}. From this, we propose that LBGs are \textit{game experiences} that connect the \textit{physical space with the virtual space} and make use of an underlying \textit{narrative} element.

This paper focuses on the integration of the terms mentioned above into the navigation between POIs in LBGs. Therefore, the following sections will provide a more detailed definition of these terms followed by an analysis of how navigation is used within hybrid LBGs that take place in cities.