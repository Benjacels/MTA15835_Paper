\subsection{Activities in Location-based Games}
In order to describe the game activities of LBGs, it is first important to look into what constitutes a game. There are a range of different definitions of games, however McGonigal proposes four defining traits of games which fit the scope of this project. Games must have a \textit{goal}, \textit{rules}, \textit{a feedback system}, and \textit{voluntary participation} \cite{RealityIsBroken}. The goal of the game is the specific outcome which players aim to achieve and what gives players a sense of purpose. The rules set limitations or remove obvious ways of getting to the goal and push players to be creative and use strategic thinking. As mentioned by McGonigal, an example of these fundamental traits can be seen in the game \textit{Scrabble}. In this game, the goal is to spell out long words with lettered tiles, while the rules are that players only have seven letters to work with at a time and they must be based on words that other players already have created. The feedback system informs players about their progress in achieving their goal e.g. through points, levels, a score, or a progress bar. This gives a promise to the player that the goal can be achieved and thereby provides motivation to keep playing. Voluntary participation requires that all players accept the goal, rules, and feedback. This establishes a common ground for the players to play together, and the freedom to enter or leave the game ensures that stressful or challenging work is experienced as a safe and pleasurable activity. 

Hybrid LBGs are designed both with the purpose of player enjoyment, by using elements from ludic LBGs, as well as educating them about e.g. cultural heritage, by using elements from pedagogic LBGs \cite{LBG_Review}. In the following, these different types of LBGs will be elaborated on, however due to the scope of this project, less emphasis will be put on purely pedagogic games.

Although the focus of ludic LBGs is enjoyment, learning is often an implicit element, since players might develop skills such as exploration and orientation e.g. by navigating a city. This is especially seen in treasure hunts, where players typically move to certain physical locations and use the physical space at the location for some interaction in the virtual space. Gentes et al. describe treasure hunts as experiences that encourage people to pay attention to details in the city and read the cityscape by looking for clues. An example of this can be seen in the LBG \textit{Team Exploration}, where players work together to compare pictures in the virtual space to real physical locations in Paris in order to figure out which areas of a map the pictures were taken at \cite{GamingOnTheMove}. The goal of the game is to reach the final location, which is shown on a map, once all pictures have been located. The limitation is that it must be done within a certain amount of time, however in the evaluation of the game, players mentioned that this limitation turned the experience more into a race, which made it difficult for players to enjoy the city instead. Gentes et al. describe this as a tension that exists in treasure hunts between the attention players allocate to the discovery of a place and the hunt itself \cite{GamingOnTheMove}. Furthermore, the evaluation showed that players wish they had some proof that they had been at certain locations, e.g. by being able to save a picture of the location in order to make the visit more meaningful. As these pictures would act as proof for progression, this indicates that the ability to save information about the places visited is a fitting way of incorporating feedback systems into treasure hunts. Treasure hunts also typically allow players to collect virtual objects at certain physical locations \cite{LBG_Review}, such as in \textit{Insectopia}, where the players collect virtual insects, which represent points and act as both the goal of the game as well as an indication of progression and feedback system \cite{Insectopia}.

Pedagogic games explicitly have the purpose of educating the player through informal learning \cite{LBG_Review}. Informal learning is learning that typically does not take place in classrooms, is not highly structured, and where the control of learning rests in the hands of the learner \cite{informallearning}. Incidental learning is informal learning that occurs when people are not conscious of it, e.g. as a result of completing a specific task \cite{informallearning}. According to Avouris \& Yiannoutsou, these games typically have a strong narrative and use role playing by making players enact certain roles to comprehend complex scenarios \cite{LBG_Review}. In these games, it is assessed that it is particularly important that the physical and virtual have a strong interplay to support learning.

Hybrid LBGs are typically used at cultural heritage sites such as museums \cite{LBG_Review}. They tend to act as guides for exhibits and aim to make them more interesting. The game activities frequently incorporate a narrative, as described in detail in the next section, through role play combined with activities such as answering questions that are related to the cultural artefact in the physical space. \textit{CityTreasure} is an example of a hybrid treasure hunt LBG where learning is supported through riddles at POIs \cite{botturi2009city}. In this game, students on a field trip visit cultural heritage sites in the city Lugano and answer riddles in the virtual space related to the POIs in the physical space. The students play in groups and are guided to the POIs through locations on a map, and as they reach the locations, they are given three riddles related to the POI. When the riddles are answered, the students will be given a new location on the map to walk to as well as feedback in the form of points if the answer was correct. The goal of the game is to gather the most points, which is driven by competition between the different groups of students playing. Furthermore, Botturia et al. reported that the game fostered collaboration within the groups to solve riddles \cite{botturi2009city}. In opposition to \textit{Team Exploration}, there is no time limit in \textit{CityTreasure} and by rewarding players' observations of the city through points, exploration is encouraged. Although this game does not focus on role play and narrative as the majority of pedagogic and hybrid games, it still manages to incorporate knowledge of the physical space while keeping players engaged according to the evaluation of the game  \cite{botturi2009city}.

%On the basis of 8 known definitions of games, Salen and Zimmerman (2003)\cite{RulesofPlay} define games as '\textit{...a system in which players engage in an artificial conflict, defined by rules, that results in a quantifiable outcome.}'. The system is described as a set of objects that affect each other in an environment. Each object   

%THIS IS GOOD BECAUSE IT MENTIONS RIDDLES. WE SHOULD CONSIDER INCLUDING THIS!!!
%A Review of Mobile Location-based Games for Learning across Physical and Virtual Spaces page 2121:
%'Inherent in these games is the fact that some activity takes place in physical space, like moving to a specific location, inspecting artefacts, taking pictures and recording videos or sounds. At the same time, some other part of the action takes place in virtual space, such as a) players interacting with simulators producing events, b) avatars and other characters interacting with each other and with the players, c) players doing riddles and puzzles, d) players generating information in digital for associated with physical objects etc. At the same time, the game rules define a game space.'

%THE FOLLOWING SUBSECTION IS FROM BENJAMIN'S FOUNDATIONS. MAYBE SHOULDN'T BE HERE!
%\subsection{Learning in Location-based Games}
%Through a survey of 26 papers and 15 LBMGs, Avouris et
%al. categorize the games according to their purpose and find
%the main characteristics of LBMGs
%They found that LBMGs can either be ludic; focus
%on enjoyment, pedagogic; focus on learning, or hybrid;
%focus on enjoyment and learning. In the following, the use of
%game space, narrative space, physical space, and virtual space
%is described for each category of LBMGs.
%In ludic games, the goal is to engage and motivate the player.
%Although the focus is enjoyment, learning is often an implicit
%element, since players might develop skills such as exploration
%and orientation by e.g. navigating a city. Common
%genres of ludic games are treasure hunts, action games, and
%role playing games.
%In treasure hunts, players typically have to collect virtual objects
%alone or in teams and in a specific or unlimited area, e.g.
%by following GPS coordinates. Treasure hunts typically do
%not contain strong narratives and mostly focus on exploration,
%orientation and in the case of players working in teams - social
%interaction. Due to their simple nature, they are mostly
%combined with more complex situations, in which there for
%instance might be a strong narrative or educational elements.
%Action games tend to be designed for multiple players, where
%the goal for players is to gain a certain advantage over each
%other through strategic thinking and decision making. This
%is typically done by locating other players, e.g. through GPS
%coordinates or pictures of players. These games allow for
%many diverse game situations to emerge, however with no
%narrative.
%Role playing games tend to have a strong focus on narrative
%and allow players to take enact roles that are connected to the
%narrative. They are are often called Alternate Reality Games
%(ARGs) and typically played by many participants and rely
%heavily on finding physical locations through clues.
%Pedagogic games in opposition to ludic games, explicitly
%have the purpose of educating the player. These games typically
%have a strong narrative where role playing allows players
%to enact certain roles to comprehend complex scenarios.
%In these games it is assessed that it is particularly important
%that the physical and virtual have a strong interconnection to
%support learning.
%Hybrid games combine entertainment and learning and are
%typically used in the context of cultural heritage, such as museums
%or historical cities. There are different variations of
%these hybrid games. One of them is museum mobile interactive
%games. In this genre, the objective is to deliver information
%about the exhibits to the museum visitor as well as
%allow for interaction between between the exhibits. The use
%of narrative in this genre is typically limited, however the interaction
%tends to include many ludic elements. A variation of
%this genre is museum role playing games, which tend to have
%a strong narrative. A challenge of designing hybrid games is
%selecting locations or POIs (points of interest) that are rich
%enough in information to support learning as well as entertainment
%activities. Furthermore, it is important to maintain a balance between ludic and pedagogic activities, as ludic activities
%might overshadow pedagogic activities.