\subsection{Game Experience}
In order to describe the game experiences of location-based games, it is first important to look into what constitutes a game. There is a range of different definitions of games, however McGonigal, 2009 \cite{RealityIsBroken} proposes four defining traits of games which fit our definition. Games must have a \textit{goal}, \textit{rules}, \textit{a feedback system}, and \textit{voluntary participation}. The goal of the game is the specific outcome which players aim to achieve and what gives players a sense of purpose. The rules limit or remove obvious ways of getting to the goal and push players to be creative and use strategic thinking. The feedback system informs players about their progress in achieving their goal e.g. through points, levels, a score, or a progress bar. This gives a promise to the player that the goal can be achieved and thereby provides motivation to keep playing. Voluntary participation requires that all players accept the goal, rules, and feedback. This establishes a common ground for the players to play together, and the freedom to enter or leave the game ensures that stressful or challenging work is experienced as a safe and pleasurable activity. McGonigal, 2009 \cite{RealityIsBroken} further uses the following definition from Suits (2005) \cite{grasshopper} to define games: \emph{'Playing a game is the voluntary attempt to overcome unnecessary obstacles'}. In relation to the traits previously mentioned, this definition primarily focuses on the goal, rules, and voluntary participation of a game. 

In relation to location-based games, the previous traits and definition can be seen in both the physical and virtual spaces.

%as well as the attempt are supported by both the physical and virtual spaces and together create what is known as the \textit{game space} \cite{LBG_Review}. The game space   

%On the basis of 8 known definitions of games, Salen and Zimmerman (2003)\cite{RulesofPlay} define games as '\textit{...a system in which players engage in an artificial conflict, defined by rules, that results in a quantifiable outcome.}'. The system is described as a set of objects that affect each other in an environment. Each object   

%THIS IS GOOD BECAUSE IT MENTIONS RIDDLES. WE SHOULD CONSIDER INCLUDING THIS!!!
%A Review of Mobile Location-based Games for Learning across Physical and Virtual Spaces page 2121:
%'Inherent in these games is the fact that some activity takes place in physical space, like moving to a specific location, inspecting artefacts, taking pictures and recording videos or sounds. At the same time, some other part of the action takes place in virtual space, such as a) players interacting with simulators producing events, b) avatars and other characters interacting with each other and with the players, c) players doing riddles and puzzles, d) players generating information in digital for associated with physical objects etc. At the same time, the game rules define a game space.'

\subsection{Play}

%THE FOLLOWING SUBSECTION IS FROM BENJAMIN'S FOUNDATIONS. MAYBE SHOULDN'T BE HERE!
%\subsection{Learning in Location-based Games}
%Through a survey of 26 papers and 15 LBMGs, Avouris et
%al. categorize the games according to their purpose and find
%the main characteristics of LBMGs
%They found that LBMGs can either be ludic; focus
%on enjoyment, pedagogic; focus on learning, or hybrid;
%focus on enjoyment and learning. In the following, the use of
%game space, narrative space, physical space, and virtual space
%is described for each category of LBMGs.
%In ludic games, the goal is to engage and motivate the player.
%Although the focus is enjoyment, learning is often an implicit
%element, since players might develop skills such as exploration
%and orientation by e.g. navigating a city. Common
%genres of ludic games are treasure hunts, action games, and
%role playing games.
%In treasure hunts, players typically have to collect virtual objects
%alone or in teams and in a specific or unlimited area, e.g.
%by following GPS coordinates. Treasure hunts typically do
%not contain strong narratives and mostly focus on exploration,
%orientation and in the case of players working in teams - social
%interaction. Due to their simple nature, they are mostly
%combined with more complex situations, in which there for
%instance might be a strong narrative or educational elements.
%Action games tend to be designed for multiple players, where
%the goal for players is to gain a certain advantage over each
%other through strategic thinking and decision making. This
%is typically done by locating other players, e.g. through GPS
%coordinates or pictures of players. These games allow for
%many diverse game situations to emerge, however with no
%narrative.
%Role playing games tend to have a strong focus on narrative
%and allow players to take enact roles that are connected to the
%narrative. They are are often called Alternate Reality Games
%(ARGs) and typically played by many participants and rely
%heavily on finding physical locations through clues.
%Pedagogic games in opposition to ludic games, explicitly
%have the purpose of educating the player. These games typically
%have a strong narrative where role playing allows players
%to enact certain roles to comprehend complex scenarios.
%In these games it is assessed that it is particularly important
%that the physical and virtual have a strong interconnection to
%support learning.
%Hybrid games combine entertainment and learning and are
%typically used in the context of cultural heritage, such as museums
%or historical cities. There are different variations of
%these hybrid games. One of them is museum mobile interactive
%games. In this genre, the objective is to deliver information
%about the exhibits to the museum visitor as well as
%allow for interaction between between the exhibits. The use
%of narrative in this genre is typically limited, however the interaction
%tends to include many ludic elements. A variation of
%this genre is museum role playing games, which tend to have
%a strong narrative. A challenge of designing hybrid games is
%selecting locations or POIs (points of interest) that are rich
%enough in information to support learning as well as entertainment
%activities. Furthermore, it is important to maintain a balance between ludic and pedagogic activities, as ludic activities
%might overshadow pedagogic activities.