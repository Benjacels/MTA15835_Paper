\subsection{Game Elements}
In order to understand what the game elements of a location-based game are, it is first important to look into what makes a game. 

Distinction between ludus and paida (pervasive game design/Rules of Play, chapter 22, page 2)

//In game design literature, a distinction between games and play is frequently made when defining what a game is (REF TO rules of play). 

-> Typical interactions in Location-based games Review, page 2121.

-> As LBG generally use game elements, it is necessary to investigate what these elements constitute for games in general. (Rules of Play, chapter 22, page 8)/Reality is Broken, page 21

-> Distinction between ludic and paedagogic

-> Examples of how these elements are used in LBG

A Review of Mobile Location-based Games for Learning across Physical and Virtual Spaces page 2121:
“At the same time, some other part of the action takes place in virtual space, such as a) players interacting with simulators producing events, b) avatars and other characters interacting with each other and with the players, c) players doing riddles and puzzles, d) players generating information in digital for associated with physical objects etc. At the same time, the game rules define a game space.”

Rules of Play, chapter 22, page 6

THE FOLLOWING SUBSECTION IS FROM BENJAMIN'S FOUNDATIONS. MAYBE SHOULDN'T BE HERE!
\subsection{Learning in Location-based Games}
Through a survey of 26 papers and 15 LBMGs, Avouris et
al. categorize the games according to their purpose and find
the main characteristics of LBMGs
They found that LBMGs can either be ludic; focus
on enjoyment, pedagogic; focus on learning, or hybrid;
focus on enjoyment and learning. In the following, the use of
game space, narrative space, physical space, and virtual space
is described for each category of LBMGs.
In ludic games, the goal is to engage and motivate the player.
Although the focus is enjoyment, learning is often an implicit
element, since players might develop skills such as exploration
and orientation by e.g. navigating a city. Common
genres of ludic games are treasure hunts, action games, and
role playing games.
In treasure hunts, players typically have to collect virtual objects
alone or in teams and in a specific or unlimited area, e.g.
by following GPS coordinates. Treasure hunts typically do
not contain strong narratives and mostly focus on exploration,
orientation and in the case of players working in teams - social
interaction. Due to their simple nature, they are mostly
combined with more complex situations, in which there for
instance might be a strong narrative or educational elements.
Action games tend to be designed for multiple players, where
the goal for players is to gain a certain advantage over each
other through strategic thinking and decision making. This
is typically done by locating other players, e.g. through GPS
coordinates or pictures of players. These games allow for
many diverse game situations to emerge, however with no
narrative.
Role playing games tend to have a strong focus on narrative
and allow players to take enact roles that are connected to the
narrative. They are are often called Alternate Reality Games
(ARGs) and typically played by many participants and rely
heavily on finding physical locations through clues.
Pedagogic games in opposition to ludic games, explicitly
have the purpose of educating the player. These games typically
have a strong narrative where role playing allows players
to enact certain roles to comprehend complex scenarios.
In these games it is assessed that it is particularly important
that the physical and virtual have a strong interconnection to
support learning.
Hybrid games combine entertainment and learning and are
typically used in the context of cultural heritage, such as museums
or historical cities. There are different variations of
these hybrid games. One of them is museum mobile interactive
games. In this genre, the objective is to deliver information
about the exhibits to the museum visitor as well as
allow for interaction between between the exhibits. The use
of narrative in this genre is typically limited, however the interaction
tends to include many ludic elements. A variation of
this genre is museum role playing games, which tend to have
a strong narrative. A challenge of designing hybrid games is
selecting locations or POIs (points of interest) that are rich
enough in information to support learning as well as entertainment
activities. Furthermore, it is important to maintain a balance between ludic and pedagogic activities, as ludic activities
might overshadow pedagogic activities.