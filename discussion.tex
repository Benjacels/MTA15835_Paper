\section{Discussion}
%One of the main objectives of this study was to design a navigational method in which an interplay between the physical and virtual world occurs. We proposed a navigational method that incorporated location-based game activities, where participants navigated by solving riddles in the virtual space about landmarks in the physical space. 
Results from our study clearly shows that riddle solving is a more enjoyable way of navigating than a 2D digital map. However, findings from interviews with the families revealed that some children also enjoyed navigating with maps. Children were challenged using the map as well as riddles for navigation, perhaps because they are not used to any of the navigational methods. We found that children thought they were better at navigating with riddles than maps. Similarly, parents were significantly more in flow with riddle-based navigation than with the 2D map, meaning there was a better challenge-skill balance with riddle-based navigation. One of the explanations could be that parents were less challenged by using maps, as they are used to navigate with maps, while riddle-based navigation was just as novel an experience for the parents as for the children. Particularly in the beginning, we observed that our test participants experienced difficulties understanding the rules of the riddle-based navigation, which was supported by the fact that participants spent most time on the first riddle. Questionnaire data also  supported this, as riddles scored lower on clear goals. We assume that maps were less challenging to use, despite of the GPS problems that occured during the experiment, and therefore that challenge was one of the reasons why riddle-based navigation scored higher on enjoyment. Furthermore, we also found that children enjoyed answering questions, getting feedback. This supports that incorporating game activities in a location-based experience like this, makes it more enjoyable for the players - even if it is more challenging and unneccessary obstacles have to be met. 

%The average completion time of solving one riddle across all sessions, reveals the first riddle as most challenging (in terms of time spent) indicating only a short learning curve and that collaboration is most needed only at the first riddle. 

We found no significant results about presence, though riddle-based navigation scored higher on making the participants aware of their surroundings. One of the main objectives of this study was to create a navigational method with a stronger interplay between the physical and virtual space. Interview data revealed different opinions on whether participants were more or less aware of the surroundings using riddle-based navigation, where landmarks are used to navigate between POIs. With map navigation, some participants felt that they were more free to notice things in the surroundings, while riddle-based navigation only made some more aware of the landmarks. This might have been due to different levels of engagement and roles that the participants took, but also the more collaborative approach some had in the families during riddle solving.   

%Others emphasized that one of the qualities of riddle-based navigation was that it made one aware of things that otherwise would have gone unnoticed.

Despite not being the main focus of this study, we observed some interesting elements in terms of social interaction among participants. Even though we did not find any statistically significant results supporting that participants helped each other more during riddle-based navigation, compared to a map, we found that riddle-based navigation has potential in motivating groups of people, making it a enjoyable group experience rather just a matter of getting from A to B, where on person is in charge. We observed that participants discussed more and that topics revolved around solving the riddles, discussing the landmarks and not on topics outside the activity, as it was the case with map navigation. Though this requires a more thorough analysis of the interaction among the participants, we hypothesize that riddle-based navigation has potential in supporting learning e.g. about landmarks or developing skills in terms of exploration, particularly in group context. Even though enjoyment is the main objective of most LBGs as mentioned by Avouris \& Yiannoutsou \cite{LBG_Review}, we recommend looking into riddle-based navigation or similar approaches for informal learning, as this study has shown the potential in incorporating the time spent navigating into the overall enjoyable experience of a location-based game. 

%However, we found no statistically significant results in terms of participants helping each other more during riddle-based navigation. 

%- Social interaction --> Potential in motivating groups of people, making it a family experience rather than a matter of getting from A to B. Potentials in terms of learning about the environment. 