\section{Introduction}
Mobile technologies are increasingly being used to create experiences in the context of museums and cities. Families and children in particular are an ideal target audience in this context, due to the rising trend of families owning mobile devices \cite{Statistik}. Previous studies concerning engaging children and families have looked into game experiences inspired by treasure hunts, where the players search for written or visual clues in order to find specific items in a museum exhibit \cite{Lynge}\cite{larsen2014tourist}. Jensen investigated, how children can be motivated to engage in a joyful museum experience, by interacting with an agent and taking pictures of art works on a tablet device \cite{Lynge}. This experience was inspired by a paper version of a treasure hunt, similar to the one investigated by Larsen \& Svabo. They investigated a treasure hunt in pamphlets, where children were dependent on their parents reading out the questions, interpreting the answers and writing them down, making it a family-activity rather than a child-activity \cite{larsen2014tourist}. We address these experiences and refer to them as \textit{mobile location-based games} (LBGs). Upscaling such experiences at museums to the city context, we did not find any studies on LBGs targeted families. Common to LBGs are that they place in a \textit{physical space} (e.g. require going to a specific physical location), require some interaction by the player in the \textit{virtual space} (e.g. solving puzzles, interacting with an avatar or following a map), resulting in an interplay between the physical and virtual space \cite{LBG_Review}. %This interplay between physical and virtual space also applies to experiences in museums or cities. Players navigate between points of interests (POIs), A and B, in the physical space and a mobile device is used as either (1) aid to get from A to B (e.g. using a digital map to navigate from one exhibit or cultural heritage to another) or (2) for some activity at the POIs (e.g. getting information, interacting with an artefact or taking pictures). 
From previous research, we found that a common tendency for LBGs is that the player simply uses either a physical or a digital map utilizing the Global Positioning System (GPS) for navigating between A and B, points of interest (POIs). Since the purpose of LBGs is to create enjoyable experiences by creating an interplay between the physical and the virtual world, we hypothesize that a navigational method with LBG activities in the navigation instead of a map can increase enjoyment of the experience. In order to evaluate the effects of such navigational method, we designed and implemented a location-based game, \textit{Lost on Earth}, aimed at families. The game was based on the previously mentioned museum experience by Jensen \cite{Lynge}, targeting 9-11 years old children. In \textit{Lost on Earth}, families navigate between POIs using riddle solving as navigational method, which we compared with a 2D digital map. In the following paper, we describe the design end evaluation of this game. 

%In this context, we found several LBGs for other target groups, where the mobile device is used for interaction at points of interest (POIs), similar to those in museums, e.g. getting information about artefacts, interacting with them or taking pictures as typical behaviours of tourists. 

%Tourists navigate between points of interests (POIs), A and B, in the physical space (See Figure \ref{fig:overview}) and a mobile device is used as either (1) aid to get from A to B (e.g. using a digital map to navigate from one exhibit or cultural heritage to another) or (2) for some activity at the POIs (e.g. getting information, interacting with an artefact or taking pictures). 

%Mobile devices are an ideal platform to use in this context, because they are increasingly becoming popular among families. 

%LBGs are interesting in the tourist domain, as they have the unique ability to captivate and entertain its audience by creating an interplay between the virtual and physical world, encouraging the user to engage with the environment. In the following paper, we investigate how navigating by solving riddles affects the experience of a LBG experience. 

%We found that existing research on location-based games primarily deal with the local interactions at points of interest (e.g. cultural heritages), rather than on what happens in between. Only few location-based games use methods other than maps to guide players. The following study is an attempt to address this problem by evaluating the effects of a location-based game experience for tourist families, where the players are guided to points of interest without the use of map opposed to the same game experience with the use of a digital map.
 