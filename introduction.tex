\section{Introduction}

Mobile technologies are increasingly being used to create experiences in the context of museums and cities. Previous studies concerning engaging children and families have looked into game experiences inspired by treasure hunts, where the players search for written or visual clues in order to find specific items in a museum exhibit \cite{Lynge} \cite{larsen2014tourist}. Jensen investigated, how children can be motivated to engage in a joyful museum experience, by interacting with an agent and taking pictures of art works on a tablet device \cite{Lynge}. This experience was inspired by a paper version of the treasure hunt, similar to the one investigated by Larsen \& Svabo. They investigated treasure trails in pamphlets, where children were dependent on their parents reading out the questions, interpreting the answers and writing them down, making it a family-activity rather than a child-activity \cite{larsen2014tourist}. Since much tourism is about being together and having time with one’s family \cite{larsen2014tourist}, these type of activities are often compelling for tourist families. Mobile devices are an ideal platform to use in this context, because they are increasingly becoming popular among families. In this study, we address these experiences and refer to them as \textit{mobile location-based games} (LBGs), as they make use of the physical and virtual spaces to create enjoyable game experiences. Upscaling such experiences at museums to the city context, we did not find any studies on LBGs targeted tourist families. In this context, we found several LBGs for other target groups, where the mobile device is used for interaction at points of interest (POIs), similar to those in museums, e.g. getting information about artefacts, interacting with them or taking pictures as typical behaviours of tourists. To allow tourists to find the POIs, they are often required to navigate between them. A common tendency for LBGs is that the user simply uses either a digital or physical map for navigation. Since the purpose of LBGs is to enhance physical spaces either for the purpose of enjoyment or learning, we hypothesize that using LBG activities in the navigation itself instead of a map, can increase the enjoyment or learning value.  \textit{Lost on Earth} is a LBG based on the previously mentioned museum experience by Jensen \cite{Lynge}, where families families navigate between POIs in a city. We designed a game in which players navigate by solving riddles and compared it with a 2D map. In the following paper, we describe the design end evaluation of this game. 

%LBGs are interesting in the tourist domain, as they have the unique ability to captivate and entertain its audience by creating an interplay between the virtual and physical world, encouraging the user to engage with the environment. In the following paper, we investigate how navigating by solving riddles affects the experience of a LBG experience. 

%We found that existing research on location-based games primarily deal with the local interactions at points of interest (e.g. cultural heritages), rather than on what happens in between. Only few location-based games use methods other than maps to guide players. The following study is an attempt to address this problem by evaluating the effects of a location-based game experience for tourist families, where the players are guided to points of interest without the use of map opposed to the same game experience with the use of a digital map.
 