\subsection{Narrative in Location-based Games}
Different disciplines (e.g. narratology, linguistics, literary studies, film studies and philosophy) define narrative with a great number of different characteristics\cite{Grimaldi}. A narrative can be defined as \emph{'a perceived sequence of non-randomly connected events, i.e., of described states or conditions which undergo change (into some different states of conditions)'}\cite{narrativeDef}. When looking into interactive narratives, it is important to understand the concept of player choice. The quality of a game design can be characterized by looking at the relationship between the player’s choice and the system’s response\cite{RulesofPlay}. This relationship should both be supported in terms of the feedback system of the game such as receiving points, known as \textit{discernable} relationships as well as in the larger context of the game, affecting the overall goal, where the outcome of the game should rely on players' choices, known as \textit{integrated} relationships\cite{RulesofPlay}. This can be related to interactive narratives, which offer players choices and the ability to navigate within a multi-linear branching structure of the narrative, thereby influencing the narrative\cite{ryanavatars}. Avouris \& Yiannoutsou state that a narrative in the shape of an interactive course is considered a promising direction of future LBGs\cite{LBG_Review}. To understand what characterises the quality of choice and narrative in LBGs, a review of interactive narratives in LBGs is presented in the following.
%Different disciplines (e.g. narratology, linguistics, literary studies, film studies and philosophy) define narrative with a great number of different characteristics\cite{Grimaldi}. A narrative can be defined as \emph{'a perceived sequence of non-randomly connected events, i.e., of described states or conditions which undergo change (into some different states of conditions)'}\cite{narrativeDef}. The game designers Katie Sallen \& Eric Zimmerman emphasize the importance of choice in a game when designing meaningful play, which emerges from the interaction between players and the system\cite{RulesofPlay}. Avouris \& Yiannoutsou state that a narrative in the shape of an interactive course is considered a promising direction of future LBGs\cite{LBG_Review}. An interactive narrative offers the user choices and to navigate within a multi-linear branching structure of the narrative\cite{ryanavatars}. Sallen \& Zimmerman write that meaningful play is the goal of a successful game design. The quality of a game design can be characterized by looking at the relationship between the player’s choice and the system’s response\cite{RulesofPlay}. To understand what characterises the quality of choice and narrative in a game design, LBGs using an interactive narrative are reviewed.

%Avouris \& Yiannoutsou wrote a review of Mobile LBGs for learning from fifteen studies, finding that narratives are common in LBGs\cite{LBG_Review} 
Khaled et al. highlight how an interactive narrative can be used to explore both the physical space but also the virtual space. By changing location the development of the story changes. The authors observed four test subjects and found that contrasts between the story world and real world forced the reader to pay close attention to the physical setting in order to make sense of the experience\cite{StoryTrek}. Similarly, Avouris \& Yiannoutsou found that LBGs emphasising on the narrative often have a strong interplay between the physical space and the virtual space\cite{LBG_Review}. Khaled et al. observed that when the users had a heightened awareness of both real world and story world, reflection on story contents occurred\cite{StoryTrek}. A qualitative study made by Blythe et al. investigated the enjoyability of an LBG called \textit{Riot!}, which revolves around progressing a story\cite{InterdisciplinaryCriticism}. In this game, users experience a story through sound that changes dynamically in relation to their location in a city, promoting a strong interplay between the physical and virtual spaces of the game. Results from 30 semi-structured interviews (the exact number of participants were not promoted) revealed that making blind choices caused disappointment, as users were not able to chose specific audio files to hear, since no information about the files was given. In Riot!, users are guided through the city through sound, while still maintaining a strong interplay between the physical and virtual spaces. This means that using sound as a navigational method might be a possibility for a LBG, however as the following section will reveal, it has some difficulties in the context of families.

