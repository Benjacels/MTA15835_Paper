\subsection{Narrative in Location-based Games}
Different disciplines (e.g. narratology, linguistics, literary studies, film studies and philosophy) define narrative with a great number of different characteristics\cite{Grimaldi}. A narrative can be defined as \emph{'a perceived sequence of non-randomly connected events, i.e., of described states or conditions which undergo change (into some different states of conditions)'}\cite{narrativeDef}. Avouris \& Yiannoutsou wrote a review of Mobile LBGs for learning from fifteen studies, finding that narratives are common in LBGs\cite{LBG_Review}. The game designers Katie Sallen \& Eric Zimmerman emphasize the importance of choice in a game when designing meaningful play, which emerges from the interaction between players and the system\cite{RulesofPlay}. Avouris \& Yiannoutsou state that a narrative in the shape of an interactive course is considered a promising direction of future LBG\cite{LBG_Review}. An interactive narrative offers the user choices and to navigate within a multi-linear branching structure of the narrative\cite{ryanavatars}. Sallen \& Zimmerman write that meaningful play is the goal of a successful game design. The quality of a game design can be characterized by looking at the relationship between the player’s choice and the system’s response\cite{RulesofPlay}. To understand what characterises the quality of choice and narrative in a game design, LBGs using an interactive narrative are reviewed.
 
Khaled et al. highlights how an interactive narrative can be used to explore both the physical space but also the virtual space. By changing location the development of the story changes. The authors observed four test subjects and found that contrasts between the story world and real world forced the reader to pay close attention to the physical setting in order to make sense of the experience\cite{StoryTrek}. Similarly Avouris \& Yiannoutsou found that LBGs emphasising on the narrative often have a strong interplay between the physical space and the virtual space\cite{LBG_Review}. Khaled et al. observed that when the users had a heightened awareness of both real world and story world, reflection on story contents occurred\cite{StoryTrek}. Blythe et al. highlights the study of Riot! where users explore a historical riot by changing location, which affects the narrative progression and which audio file the system plays. Results from 30 semi-structured interviews (the exact number of participants were not promoted) revealed a lack of choice caused disappointment when users could not freely discover a wanted file. The users chose which scene to hear, but no information about the scenes were given resulting in users making blind choices\cite{InterdisciplinaryCriticism}.